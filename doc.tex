\documentclass[12pt]{article}


\usepackage[ngerman]{babel}
\usepackage[utf8]{inputenc}

\usepackage{enumitem, amssymb}
\usepackage{geometry}
\usepackage{xcolor}
\usepackage{tcolorbox}
\usepackage{listings}
\usepackage[scaled=1]{inconsolata}
\usepackage[T1]{fontenc}

\newlist{checklist}{itemize}{2}
\setlist[checklist]{label=$\square$}

\usepackage{titling}
\newcommand{\subtitle}[1]{%
  \posttitle{%
    \par\end{center}
    \begin{center}\large#1\end{center}
    \vskip0.5em}%
}

\definecolor{commentColor}{HTML}{adb5bd}
\definecolor{mygray}{rgb}{0.5,0.5,0.5}
\definecolor{stringColor}{HTML}{7048e8}
\definecolor{keywordColor}{HTML}{228be6}
\definecolor{backgroundColor}{HTML}{f8f9fa}
\definecolor{borderColor}{HTML}{f1f3f5}
\definecolor{inlineTextColor}{HTML}{495057}
\definecolor{linkcolor}{HTML}{228be6}

\lstset{
  backgroundcolor=\color{backgroundColor},   % choose the background color; you must add \usepackage{color} or \usepackage{xcolor}; should come as last argument
  basicstyle=\ttfamily,        % the size of the fonts that are used for the code
  breaklines=true,                 % sets automatic line breaking
  commentstyle=\color{commentColor},    % comment style
  extendedchars=true,              % lets you use non-ASCII characters; for 8-bits encodings only, does not work with UTF-8
  firstnumber=1000,                % start line enumeration with line 1000
  frame=single,	                   % adds a frame around the code
  keepspaces=true,                 % keeps spaces in text, useful for keeping indentation of code (possibly needs columns=flexible)
  keywordstyle=\color{keywordColor}\textbf,       % keyword style
  language=Python,                 % the language of the code
  morekeywords={*,...},            % if you want to add more keywords to the set
  numbers=none,                    % where to put the line-numbers; possible values are (none, left, right)
  numbersep=5pt,                   % how far the line-numbers are from the code
  numberstyle=\tiny\color{mygray}, % the style that is used for the line-numbers
  rulecolor=\color{borderColor},         % if not set, the frame-color may be changed on line-breaks within not-black text (e.g. comments (green here))
  showstringspaces=false,          % underline spaces within strings only
  showtabs=false,                  % show tabs within strings adding particular underscores
  stepnumber=2,                    % the step between two line-numbers. If it's 1, each line will be numbered
  stringstyle=\color{stringColor},     % string literal style
  tabsize=2
}

\usepackage{hyperref}
\hypersetup{
    colorlinks,
    citecolor=black,
    filecolor=black,
    linkcolor=commentColor,
    urlcolor=linkcolor
}

\geometry{
  a4paper,
  left=33mm,
  right=33mm,
  top=20mm
}

\definecolor{mordantred19}{rgb}{0.68, 0.05, 0.0}

\newcommand{\solution}{\textcolor{mordantred19}{Solution:}}

\newcommand{\hinweis}{\textcolor{stringColor}{! Hinweis: }}

\let\oldemph\emph
\renewcommand{\emph}[1]{\textbf{\oldemph{#1}}}

%\let\oldsection\section
%\renewcommand{\section}[1]{
%  \begin{center}
%  \oldsection{#1}
%  \hrule
%  \end{center}
%}

\renewcommand{\familydefault}{\sfdefault}

\newcommand{\schritt}[3]{
\begin{tcolorbox}
  \textbf{Schritt #1: #2}
\end{tcolorbox}
#3
}


\title{Malroboter mit dem Raspberry Pi}
\subtitle{Filmbüro Wismar}
\author{Richard Grünert und Josefine Richey}
\date{12/2020}

\begin{document}

\maketitle

\tableofcontents

\pagebreak

\section{Kurzbeschreibung}
Diese Anleitung dient dem Aufbau sowie dem Einsatz eines einfachen Roboters in einem KURSKURSKURS. Bei diesem Roboter werden zwei Motoren mithilfe eines \textsc{Raspberry-Pi}-Einplatinencomputers angesteuert, um einen Stift über ein  Papierblatt zu bewegen. Die Geschwindigkeit beider Motoren kann dabei durch einfache Programmierbefehle eingestellt werden.\\

Die ursprüngliche Idee des Projektes stammt \href{https://tuduu.org/projekt/automatischer-malroboter}{hierher}\footnote[1]{https://tuduu.org/projekt/automatischer-malroboter}. Da kein \textsc{Calliope} im Filbüro vorhanden war, wurde dieses Projekt auf einem \textsc{Raspberry Pi} umgesetzt, was einige Zwischenschritte erforderte.\\

\subsection{Ziel}
Das Projekt soll den Kursteilnehmern die Möglichkeiten der Programmierung einfacher Elektronik nahebringen.

\section{Vorbereitung}
Je nach Durchführung sowie verfügbarer Zeit sollten einige Schritte des Aufbaus im Vornherein erledigt werden, z.B. die Verdrahtung der Motoren (später beschrieben).\\

\subsection{Materialien}

\subsubsection{Werkzeuge}
\begin{checklist}
    \item Heißklebepistole
    \item Schere / Teppichmesser
    \item evtl. Abisolierzange
\end{checklist}

\subsubsection{Material}
\begin{checklist}
    \item RaspberryPi
    \item Bildschirm + HDMI-Kabel
    \item Maus
    \item Tastatur
    \item Zahnstocher
    \item Motoren
    \item Treiber
    \item Pappe
    \item Kabel
    \item Räder
    \item Klammern
    \item 4-AA-Batteriefach
    \item 4 AA Batterien
    \item Unterlage (Telefonbuch / Karton)
    \item X Female-Female Jumperkabel
    \item 1x Female-Male Jumperkabel
\end{checklist}


\subsection{Software}
Auf dem RaspberryPi sollte bereits ein funktionierendes \emph{Betriebssystem} installiert sein. Es bietet sich \emph{Raspberry Pi OS} (Raspbian) an. \\

Bei dem ursprünglichen Projekt wurde die Programmierumgebung \emph{Scratch} verwendet. Da diese die Ansteuerung von Motoren mit dem Raspberry Pi jedoch unnötig kompliziert macht, wurde hier auf die Verwendung der Programmiersprache \emph{Python} ausgewichen. Python bietet zwar keine grafische Programmieroberfläche (wie Scratch), ist jedoch, nach Meinung der Autoren, eine für alle Altersgruppen intuitive und mindestens genauso zugängliche Programmiermethode wie Scratch.\\

Vorkenntnisse über Python sind nicht zwingend notwendig!

\subsubsection{Script}
Zur Vorbereitung sollten einige Funktionalitäten innerhalb von Python eingebaut werden, welche die Verwendung bei der Kursdurchführung für die Kursteilnehmer erleichtern. Die Nachfolgenden Programme sollten vorbereitet werden.\\

Führen Sie folgende Schritte durch, sofern die genannten Ordner und Dateien nicht bereits vorhanden sind:\\

\schritt{1}{Legen Sie einen Projektordner an}{

  Legen Sie auf dem Desktop des Raspberry Pi einen neuen Ordner an und nennen Sie ihn \emph{Malroboter}.\\

}

\schritt{2}{Erstellen Sie zwei neue Dateien}{

  Erstellen Sie im Ordner Malroboter die folgenden Dateien:
  \begin{itemize}
  \item \emph{motorsteuerung.py}: Diese Datei dient der Übersetzung der bestehenden Funktionen und der Definitionen innerhalb des Projektes und wird i.d.R. nicht von den Kursteilnehmern verwendet.
  \item \emph{kurs.py}: Diese Datei soll dann von den Kursteilnehmern programmiert werden, hier wird z.B. die Geschwindigkeit der Motoren eingestellt. 
  \end{itemize}


}

\schritt{3}{Fügen Sie das Steuerungsprogramm ein}{

 Fügen Sie den folgenden Programmtext in die Datei \emph{motorsteuerung.py} ein:\\
  \indent \hinweis Sie finden die Programmdateien auch \href{https://www.github.com/latenighticecream/malroboter/Malroboter}{hier}\footnote[2]{https://www.github.com/latenighticecream/malroboter/Malroboter}.\\

}

\emph{motorsteuerung.py}\\
\lstinputlisting{Malroboter/motorsteuerung.py}
\vspace{\baselineskip}

\schritt{4}{Binden Sie das Steuerungsprogramm ein}{
Öffnen Sie die Datei \emph{kurs.py} und fügen sie den folgenden Programmtext ein:
\lstinputlisting{Malroboter/kurs.py}
}


\section{Durchführung}

\subsection{Bau des Roboterarmes}
\schritt{1}{Den Roboterarm vorbereiten}{
Die Pappe schneidet ihr in sechs 30cm lange und 2cm breite Streifen.
Zwei davon werden auf 15cm gekürzt, zwei weitere auf 12cm.\\

Aus den Resten der gekürzten Streifen schneidet ihr zirka 12 kleine Pappquadrate.
Außerdem benötigt ihr vier Pappteile, die ihr wie auf dem Bild zuschneidet, und zwei Pappkreise, deren Durchmesser mit der Größe der Reifen übereinstimmt.\\
}


\schritt{2}{Den Roboterarm bauen}{
Die Pappstreifen werden jetzt zusammengeklebt.
Dafür klebt ihr die jeweils beiden gleichlangen Streifen mit Heißkleber zusammen, sodass ihr nun ein 30cm, ein 15cm und ein 12cm Pappstück habt. 
Die vier Pappteile klebt ihr übereinander zu einem Klotz. An diesem können später die Stifte befestigt werden.\\
}


\subsection{Verbindung mit RaspberryPi}

\subsubsection{Verbindung Motor mit Treiber}
\subsubsection{Verbindung Treiber mit RaspberryPi}

\subsection{Programmierung}
\subsection{Programm A (Beispiel)}
\subsection{Programm B}


\section{Projektvariation}
je nach Altersklasse

\section{Problembehandlung}


\end{document}
