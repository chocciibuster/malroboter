\documentclass[12pt]{article}



\usepackage[ngerman]{babel}
\usepackage{enumitem, amssymb}
\usepackage{geometry}
\usepackage{xcolor}
\usepackage{tcolorbox}

\newlist{checklist}{itemize}{2}
\setlist[checklist]{label=$\square$}

\usepackage{titling}
\newcommand{\subtitle}[1]{%
  \posttitle{%
    \par\end{center}
    \begin{center}\large#1\end{center}
    \vskip0.5em}%
}

\usepackage{hyperref}
\hypersetup{
    colorlinks,
    citecolor=black,
    filecolor=black,
    linkcolor=black,
    urlcolor=black
}

\geometry{
  a4paper,
  left=33mm,
  right=33mm,
  top=20mm
}

\definecolor{mordantred19}{rgb}{0.68, 0.05, 0.0}

\newcommand{\solution}{\textcolor{mordantred19}{Solution:}}

\let\oldemph\emph
\renewcommand{\emph}[1]{\textbf{\oldemph{#1}}}

%\let\oldsection\section
%\renewcommand{\section}[1]{
%  \begin{center}
%  \oldsection{#1}
%  \hrule
%  \end{center}
%}

\renewcommand{\familydefault}{\sfdefault}

\newcommand{\schritt}[3]{
\begin{tcolorbox}
  \textbf{Schritt #1: #2}
\end{tcolorbox}
#3
}


\title{Malroboter mit dem Raspberry Pi}
\subtitle{Filmbüro Wismar}
\author{Richard Grünert und Josefine Richey}
\date{12/2020}

\begin{document}

\maketitle

\tableofcontents

\pagebreak

\section{Kurzbeschreibung}
Diese Anleitung dient dem Aufbau sowie dem Einsatz eines einfachen Roboters in einem KURSKURSKURS. Bei diesem Roboter werden zwei Motoren mithilfe eines \emph{Raspberry-Pi}-Einplatinencomputers angesteuert, um einen Stift über ein  Papierblatt zu bewegen. Die Geschwindigkeit beider Motoren kann dabei durch einfache Programmierbefehle eingestellt werden.\\

\subsection{Ziel}
Das Projekt soll den Kursteilnehmern die Möglichkeiten der Programmierung einfacher Elektronik nahebringen.

\section{Vorbereitung}
Je nach Durchführung sowie verfügbarer Zeit sollten einige Schritte des Aufbaus im Vornherein erledigt werden, z.B. die Verdrahtung der Motoren (später beschrieben).\\

\subsection{Materialien}

\subsubsection{Werkzeuge}
\begin{checklist}
    \item Heißklebepistole
    \item Schere
\end{checklist}

\subsubsection{Material}
\begin{checklist}
    \item RaspberryPi
    \item Zahnstocher
    \item Motoren
    \item Treiber
    \item Pappe
    \item Kabel
    \item Räder
    \item Klammern
    \item Unterlage (Telefonbuch / Karton)
\end{checklist}


\subsection{Software}
Auf dem RaspberryPi sollte bereits ein funktionierendes \emph{Betriebssystem} installiert sein. Es bietet sich \emph{Raspberry Pi OS} (Raspbian) an. \\

Bei dem ursprünglichen Projekt wurde die Programmierumgebung \emph{Scratch} verwendet. Da diese die Ansteuerung von Motoren mit dem Raspberry Pi jedoch unnötig kompliziert macht, wurde hier auf die Verwendung der Programmiersprache \emph{Python} ausgewichen. Python bietet zwar keine grafische Programmieroberfläche (wie Scratch), ist jedoch, nach Meinung der Autoren, eine für alle Altersgruppen intuitive und mindestens genauso zugängliche Programmiermethode wie Scratch.\\

Vorkenntnisse über Python sind nicht zwingend notwendig!

\subsubsection{Script}
Zur Vorbereitung sollten einige Funktionalitäten innerhalb von Python eingebaut werden, welche die Verwendung bei der Kursdurchführung für die Kursteilnehmer erleichtern. Die Nachfolgenden Programme sollten vorbereitet werden.\\

\schritt{1}{Legen Sie einen Projektordner an}{

  Legen Sie auf dem Desktop des Raspberry Pi einen neuen Ordner an und nennen Sie ihn \emph{Malroboter} (sofern nicht bereits vorhanden).\\

}

\schritt{2}{Erstellen Sie eine neue Datei}{

  Erstellen Sie im Ordner Malroboter eine neue Datei \emph{motorsteuerung.py} (sofern nicht bereits vorhanden).\\

}

\schritt{3}{Fügen Sie den folgenden}{2}


\schritt{4}{Einbinden des Programms}{3}


\section{Durchführung}

\subsection{Bau des Roboterarmes}

\subsection{Verbindung mit RaspberryPi}

\subsubsection{Verbindung Motor mit Treiber}
\subsubsection{Verbindung Treiber mit RaspberryPi}

\subsection{Programmierung}
\subsection{Programm A (Beispiel)}
\subsection{Programm B}


\section{Projektvariation}
je nach Altersklasse


\end{document}
