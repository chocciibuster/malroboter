\usepackage[ngerman]{babel}
\usepackage[utf8]{inputenc}
\usepackage{float}
\usepackage{booktabs}
\usepackage{enumitem, amssymb}
\usepackage{geometry}
\usepackage{xcolor}
\usepackage{tcolorbox}
\usepackage{listings}
\usepackage{graphicx}
\usepackage[scaled=1]{inconsolata}
\usepackage{fontspec}
\usepackage{subfig}

\graphicspath{{./graphics/}}

\newlist{checklist}{itemize}{2}
\setlist[checklist]{label=$\square$}

\usepackage{titling}
\newcommand{\subtitle}[1]{%
  \posttitle{%
    \par\end{center}
    \begin{center}\large#1\end{center}
    \vskip0.5em}%
}

\definecolor{commentColor}{HTML}{adb5bd}
\definecolor{mygray}{rgb}{0.5,0.5,0.5}
\definecolor{stringColor}{HTML}{7048e8}
\definecolor{keywordColor}{HTML}{228be6}
\definecolor{backgroundColor}{HTML}{f8f9fa}
\definecolor{borderColor}{HTML}{f1f3f5}
\definecolor{inlineTextColor}{HTML}{495057}
\definecolor{linkcolor}{HTML}{228be6}
\definecolor{yelluw0}{HTML}{ffec99}
\definecolor{yelluw}{HTML}{ffe066}
\definecolor{yelluw2}{HTML}{fab005}
\definecolor{yelluw3}{HTML}{f08c00}
\definecolor{limeu}{HTML}{d8f5a2}

\lstset{
  backgroundcolor=\color{backgroundColor},   % choose the background color; you must add \usepackage{color} or \usepackage{xcolor}; should come as last argument
  basicstyle=\ttfamily,        % the size of the fonts that are used for the code
  breaklines=true,                 % sets automatic line breaking
  commentstyle=\color{commentColor},    % comment style
  extendedchars=true,              % lets you use non-ASCII characters; for 8-bits encodings only, does not work with UTF-8
  firstnumber=1000,                % start line enumeration with line 1000
  frame=single,	                   % adds a frame around the code
  keepspaces=true,                 % keeps spaces in text, useful for keeping indentation of code (possibly needs columns=flexible)
  keywordstyle=\color{keywordColor}\textbf,       % keyword style
  language=Python,                 % the language of the code
  morekeywords={*,...},            % if you want to add more keywords to the set
  numbers=none,                    % where to put the line-numbers; possible values are (none, left, right)
  numbersep=5pt,                   % how far the line-numbers are from the code
  numberstyle=\tiny\color{mygray}, % the style that is used for the line-numbers
  rulecolor=\color{borderColor},         % if not set, the frame-color may be changed on line-breaks within not-black text (e.g. comments (green here))
  showstringspaces=false,          % underline spaces within strings only
  showtabs=false,                  % show tabs within strings adding particular underscores
  stepnumber=2,                    % the step between two line-numbers. If it's 1, each line will be numbered
  stringstyle=\color{stringColor},     % string literal style
  tabsize=2
}

\usepackage{hyperref}
\hypersetup{
    colorlinks,
    citecolor=black,
    filecolor=black,
    linkcolor=yelluw3,
    urlcolor=yelluw3
}

\geometry{
  a4paper,
  left=33mm,
  right=33mm,
  top=20mm
}

\definecolor{mordantred19}{rgb}{0.68, 0.05, 0.0}

\newcommand{\solution}{\textcolor{mordantred19}{Solution:}}

\newcommand{\hinweis}{\textcolor{stringColor}{! Hinweis: }}

\let\oldemph\emph
\renewcommand{\emph}[1]{\textbf{\oldemph{#1}}}

%\let\oldsection\section
%\renewcommand{\section}[1]{
%  \begin{center}
%  \oldsection{#1}
%  \hrule
%  \end{center}
%}

\renewcommand{\familydefault}{\sfdefault}

\newcommand{\schritt}[3]{
\begin{tcolorbox}
  \textbf{Schritt #1: #2}
\end{tcolorbox}
#3
}

\tcbset{
  colback=yelluw0,
  colframe=yelluw0
}
